\documentclass[12pt]{article}

\usepackage{mathtools}
\usepackage{advdate}
\usepackage{graphpap}
\usepackage{amsmath}
\usepackage{anysize}
\usepackage{amsthm}
\usepackage{amssymb}
\usepackage{cite}
\usepackage{graphicx}
\usepackage{mathrsfs}
\usepackage{upgreek}
\usepackage{extsizes}
\usepackage[margin=1in,footnotesep=1cm]{geometry}
\usepackage[hang]{footmisc}
\usepackage{url}

\begin{document}
	\urlstyle{same}
	\setlength\footnotemargin{10pt}

	\begin{flushleft}
	\textbf{\Large Nash Equilibrium In Quantum Game Theory}	
	
	\medskip	

	By Alex Nikanov

	\bigskip

	\textbf{\large Historical background}

	\medskip	

	Game theory, formally, is the study of mathematical models of strategic interactions amond rational decision-makers.\footnote{Myerson, Roger B. (1991). Game Theory: Analysis of Conflict, Harvard University Press, p. 1. Chapter-preview links, pp. vii–xi, \url{https://books.google.com/books?id=E8WQFRCsNr0C&printsec=find&pg=PR7#v=onepage&q&f=false}} 
	In non-technical words, game theory is a field of mathematics that studies strategies and human interactions from the mathematical point of view. It is often a valuable skills in economic sphere, and it is also used in computer science and some social sciences. 

	\bigskip

	\textbf{\large Definitions and technical background}
	
	\medskip

	A \textit{game} is defined as \textbf{THAT BOOK - GET SOME USEFUL DEFINITIONS FROM THERE ABOUT GAMES}. Games can be cooperative/non-cooperative, symmetric/asymmetric, zero-sum/non-zero-sum, and sequential/simultaneous. The definitions are as follows. In addition, there are also combinatorial, infinitely long, continuous, differential, and quantum games, as well as metagames and pooling games, which we will not concern ourselves with in this paper.

	\medskip
	A \textit{cooperative} game is 
	\\A \textit{non-cooperative} game is
	
	\medskip
	A \textit{symmetric} game is
	\\A \textit{asymmetric} game is 

	\medskip
	A \textit{zero-sum} game is 
	\\A \textit{non-zero-sum} game is
	
	\medskip
	A \textit{sequential} game is 
	\\A \textit{simultaneous} game is

	\bigskip

	\textbf{\Large Nash equlibrium in classical game theory} 
	\medskip\\Nash equilibrium is a proposed solution of a non-cooperative game involving two or more players in which each player is assumed to know the equilibrium strategies of the other players, and no player has anything to gain by changing only their own strategy.\footnote{Osborne, Martin J.; Rubinstein, Ariel (12 Jul 1994). A Course in Game Theory. Cambridge, MA: MIT. p. 14. ISBN 9780262150415.} Formally defined, \textbf{INSERT NASH EQUILIBRIUM DEFINITION HERE}. In simpler words, Alice and Bob\footnote{It is pretty common to denote players in game theory as Alice and Bob (player A and player B, respectively).}, are in Nash equilibrium if Alice is making the best decision she can, taking into account Bob's decision given that his decision is static, and Bob is making the best decision he can give that Alice's decision is static. A strategy in Nash equilibrium is a best response to all the other strategies in that equilibrium, as no player can benefit from changing his or her strategy. If generalized for more than two players, a group of players are in Nash equilibrium if each one is making best decision possible, taking into account the decisions of all other playes, as long as the other parties' decisions remain unchanged.\footnote{\url{https://en.wikipedia.org/wiki/Nash_equilibrium}} 

	\textbf{A LITTLE BIT OF HISTORY OF NASH EQULIBRIUM}

	\bigskip
	
	\textbf{\Large Quantum Game Theory}

	\end{flushleft}

\end{document}