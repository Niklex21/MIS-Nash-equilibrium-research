\documentclass[12pt]{article}

\usepackage{mathtools}
\usepackage{advdate}
\usepackage{graphpap}
\usepackage{amsmath}
\usepackage{anysize}
\usepackage{amsthm}
\usepackage{amssymb}
\usepackage{cite}
\usepackage{graphicx}
\usepackage{mathrsfs}
\usepackage{upgreek}
\usepackage{extsizes}
\usepackage[margin=1in]{geometry}

\begin{document}
	
	\textbf{\large Combinatorics Homework \#2}	

	\textbf{Due Tuesday, February 5, 2019}

	\medskip
	\noindent 1.) An urn contains 6 white balls and 9 red balls. If four balls are to be randomly selected without replacement, what is the probability that the first two selected are white and the last two are red?

	\medskip
	\noindent 2.) Consider three different urns. Urn one contains 2 white balls and 4 red balls; Urn Two contains 8 white balls and 4 red balls; and Urn Three contains 1 white ball and 3 red balls. If one ball is selected from each urn, what is the probability that the ball chosen from Urn One was white, given that exactly two white balls were chosen?	

	\medskip
	\noindent 3.) Apparently, Cthulhu wriets mostly in the Old Tongue, which uses an alphabet containing symbols that humans are unfamiliar with. I asked the other day how one might spell "Apple Juice" in the Old Tongue and he said, "Though humans have no hope of actually pronuncing the word, it is spelled using symbols from the old Alphabet like so;\\
	$\bigstar\Theta\Theta\diamondsuit\mho\Theta\Gamma\mho$
	\begin{enumerate}
		\item[a)] How many distinct arrangements of the symbols in the Old Tongue translation of "Apple Juice" are there?
		\item[b)] What is the probability that a 3-letter word obtained at random from the symbols in the Old Tongue translation of "Apple Juice" has no $\Theta$'s?
		\item[c)] What is the probability that a 3-letter word obtained at random from the symbols in the Old Tongue translation of "Apple Juice" has at least one $\Theta$?
	\end{enumerate}

	\medskip
	\noindent 4.) Prove the following identities by writing out both sides in factorials and simplifying.
	\begin{enumerate}
		\item[a)] $\displaystyle \frac{n+1}{r+1}{n \choose r}={n+1 \choose r+1}$
		\item[b)] $\displaystyle {n \choose m}{m \choose r}={n \choose r}{n-r \choose m-r}$
	\end{enumerate}
	
	\medskip
	\noindent 5.) Prove the following identities using induction.
	\begin{enumerate}
		\item[a)] $\displaystyle {n \choose 0}+{n+1 \choose 1}+{n+2 \choose 2}+{n+3 \choose 3}+...+{n+r \choose r}={n+r+1 \choose r}$
		\item[b)] $\displaystyle {n \choose 0}-{n \choose 1}+{n \choose 2}-{n \choose 3}+...+{(-1)}^{r}{n \choose r}={(-1)}^{r}{n-1 \choose r}$
	\end{enumerate}

	\medskip
	\noindent 6.) Prove the following identities by equating coefficients in two differnt but equal expressions, or by using some other combinatorial argument.
	\begin{enumerate}
		\item[a)] $\displaystyle \sum_{r=0}^{n}{n \choose r}{m \choose k+r}={n+m \choose n+k}$
		\item[b)] $\displaystyle \sum_{r=0}^{n}{n \choose r}^{2}={2n \choose n}$
	\end{enumerate}

\end{document}